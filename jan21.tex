\subsection{Ideals}
An ideal is similar to a subalgebra, but has a stronger limiting condition. 
\begin{definition}
An \vocab{ideal} of a Lie Algebra $L$ is a subspace $I$ such that 
\[ [x,y] \in I \text{ }\forall x\in L \text{ and } y \in I. \] 
Any ideal is also a subalgebra of $L$. 
\end{definition}
This is the analog to an ideal for a ring. 

\begin{example}[Quotient Algebra]
Let $L$ be a Lie algebra, and $I \subset L$ be an ideal. Then the vector space $L/I$ is also a Lie algebra. The vectors are the \emph{cosets} 
\[ x + I = \{x+y | y \in I \} \subset L. \] We define the bracket as 
\[ [x+I, y+I] = [x+y] + I. \] As all $[x,y] \in I$, this is well defined for cosets. This Lie algebra is called the \vocab{quotient algebra}.  
\end{example}

Let $M$ and $N$ be subspaces of $L$, an arbitrary Lie algebra. Then 
\[ [M,N] = \Span\{[x,y] | x \in M, y \in N\} = \Span \{\sum_{i} [x_i,y_i] | x_i \in M, y_i \in N. \] 
In particular, if $M$ is a subalgebra, then 
\[ [M,M] \subseteq M, \] and if $M$ is an ideal, then \[ [L,M] \subseteq M. \] 

\begin{definition}
The \vocab{derived algebra} of $L$ is $[L,L]$. 
\end{definition}

We also have the following: 
\begin{lemma*}
$[L,L] \subset L$ is an ideal, which implies it is a subalgebra, and thus itself is a Lie algebra. 
\end{lemma*}
\begin{proof}
Let $I = [L,L]$, which is a subspace of $L$ by constraint. Then 
\[ [L,I] \subseteq [L,L] = I, \] as $I \subseteq L$, and we are done. 
\end{proof}

\begin{lemma*}
Let $L = \mathfrak{gl}(n)$. Then 
\[ [L,L] = \frak{sl}(n). \] 
\end{lemma*}
\begin{proof}
We have $[L,L] \subseteq \frak{sl}(n)$, as the trace of any commutator is equal to 0. \newline

Consider the basis for $\frak{sl}(n)$. The basis for $\frak{gl}(n)$ is all $E_{ij}$ such that entry $(i,j) = 1$ and all others are 0. Then the basis for $\frak{sl}(n)$ is all $E_{ij}$, along with all $E_{k-1,k-1} - E_{k,k}$ for $1 \leq k \leq n$. \newline


Recall that $E_{ij}E_{kl} = \delta_{jk}E_{il}$. Then the bracket is 
\[ [E_{ij}, E_{kl}] = \delta_{jk}E_{il} - \delta_{il}E_{kj}. \] For $i \neq j$, $[E_{ij}, E_{jj}] = E_{ij}$. Then $[E_{ij}, E_{ji}] = E_{ii} - E_{jj}$. As all basis vectors can be generated by brackets, we are done. 
\end{proof}

Recall that $\frak{n}(n)$ was the Lie algebra of all strictly upper triangular matrices, and this is a subalgebra of $\frak{t}(n)$, the Lie algebra on all triangular matrices. These are both subalgebras of $\frak{gl}(n)$. 

\begin{exercise}
Show that $\frak{n}(n) = [\frak{t}(n), \frak{t}(n)]$.
\end{exercise}

\subsection{Centers}
Recall the central element from the last lecture. This can be generalized:
\begin{definition}
The \vocab{center} of a Lie algebra $L$, $Z(L)$, is defined such that 
\[ Z(L) = \{ z \in L | [z,x] = 0 \text{ } \forall x \in L \}. \] 
\end{definition}

We have the following: 
\begin{lemma*}
$Z(L) \subset_I L$.
\end{lemma*}
\begin{proof}
Let $z_1,z_2 \in Z(L)$. Then we have 
\[ [z_1 + z_2, x] = [z_1,x] + [z_2,x] = 0. \] This implies closure under addition. 
Next, let $\lambda \in \mathbb{F}$. We have 
\[ [\lambda z, x] = \lambda[z,x] = 0, \] which implies closure under scalar multiplication. Lastly, consider the bracket. For all $z \in Z(L)$, $x \in L$, we have 
\[ [x,z] = 0 \in Z(L), \] so $Z(L)$ is also closed under the bracket, as desired. 
\end{proof}

\begin{exercise}
Show that $Z(\frak{gl}(n)) = \Span{I}$. 
\end{exercise}

\begin{example}
Find $Z(\frak{sl}(n))$. 
\end{example}
\begin{soln}
Assuming that $\ch \mathbb{F} \nmid n$, we can write 
\[ \frak{gl}(n) = \frak{sl}(n) \oplus \mathbb{F} I. \] This implies that $Z(\frak{sl}(n)) = \{0\}$. Then $A = \frac{tr(A)}{n}I \in \frak{sl}(n)$. 

If $n = 0$ in $\mathbb{F}$, then we have 
\[ Z(\frak{sl}(n)) = \mathbb{F} I. \]
\end{soln}

\subsection{Simple Lie Algebras}

\begin{definition}
A Lie algebra is called \vocab{simple} if it is nonabelian has no nonzero proper ideals. 
\end{definition}
We require that $L$ is nonabelian as if it were, it would have $[\cdot,\cdot] = 0$ for all possible brackets. 
\begin{remark*}
if $L$ is simple, then we have 
\[ Z(L) = \{0\}, [L,L]=L. \] 
\end{remark*}

Consider the following: 
\begin{proposition}
Asume $\ch \mathbb{F} \nmid n$. Then the Lie algebra $\frak{sl}(n, \mathbb{F})$ is simple.
\end{proposition}

\begin{proof}
Let $L = \frak{sl}(n, \mathbb{F})$. Let $I \subset L$ be an ideal, $I \neq \{0\}$. It suffices to show that $I=L$. 

Let 
\[ A = (a_{kl}) \in I = \sum a_{kl}E_{kl}. \] We have that 
\[ [E_{ij},A] \in I \text{ } \forall i,j  = \sum_{k,l} a_{kl} [E_{ij}, E_{kl}] = \delta_{jk}E_{il} - \delta_{il}E_{kj}. \] This sum is nonzero only when $k = j$, so we only need to sum over $l$: 
\[ \sum_{l} a_{jl}E_{il} - \sum_{k} a_{ki}E_{kj} \in I. \] Consider 
\[ [E_{ij}, [E_{ij}, A]] = \sum_l a_{jl}[E_{ij}, E_{il}] - \sum_k a_{ki}[E_{ij},E_{kj}]. \] We can rewrite this as 
\[ -\sum_l a_{jl}\delta_{li}E_{ij} - \sum_k a_{ki}\delta_{jk}E_{ij}. \] For nonzero values, we need $l=i$ and $j=k$, so we have the above equal to 
\[ a_{ji}E_{ij} - a_{ji}E_{ij} = -2a_{ji}E_{ij} \in I. \] 

If $A = (a_{(ij)}) \in I$, then $E_{ij} \in I$ whenever $i \neq j$, $a_{ij} \neq 0$. 

This implies 2 cases: 
\begin{itemize}
\item Case 1: $I$ contains some non diagonal $A$. There exists some $i \neq j$ such that $E_{ij} \in I$, so we have
\[ [E_{ki}, E_{ij}] = E_{kj} \in I \text{ } \forall k \neq j. \] We also have 
\[ [E_{jl}, E_{kkj}] = -E_{jk} \in I \text{ } \forall x \neq l \implies E_{kl} \in I \text{ } \forall k \neq l.\] This implies every offdiagonal matrices are in the ideal. Verifying the diagonal matrices is simple, as we know that 
\[ [E_{kl}, E_{lk}] = E_{kk} - E_{ll} \in I. \] Thus $I = \frak{sl}(n)$. 
\item Case 2: $I$ contains some nonzero diagonal matrix $A = \sum\limits_{k}a_{kk}E_{kk}$, where $\sum a_{kk} = 0$ (from trace). We have 
\[ [E_{ij}, A] = (a_{jj} - a_{ii}) E_{ij} \in I, \] for all $i \neq j$. Since $\ch \mathbb{F} \nmid n$, we have that $A \neq \lambda \cdot I$ for some  $\lambda \in \mathbb{F}$. Then $\exists i \neq j$ such that $a_{ii} \neq a_{jj}$. THen $E_{ij} \in I$, and we have reduced this to Case 1. 
\end{itemize}
As we have proved Case 1, we are done. 
\end{proof}


